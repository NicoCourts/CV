%%%%%%%%%%%%%%%%%%%%%%%%%%%%%%%%%%%%%%%%%
% Medium Length Graduate Curriculum Vitae
% LaTeX Template
% Version 1.1 (9/12/12)
%
% This template has been downloaded from:
% http://www.LaTeXTemplates.com
%
% Original author:
% Rensselaer Polytechnic Institute (http://www.rpi.edu/dept/arc/training/latex/resumes/)
%
% Important note:
% This template requires the res.cls file to be in the same directory as the
% .tex file. The res.cls file provides the resume style used for structuring the
% document.
%
%%%%%%%%%%%%%%%%%%%%%%%%%%%%%%%%%%%%%%%%%

%----------------------------------------------------------------------------------------
%	PACKAGES AND OTHER DOCUMENT CONFIGURATIONS
%----------------------------------------------------------------------------------------

\documentclass[margin]{res} % Use the res.cls style, the font size can be changed to 11pt or 12pt here

\usepackage{helvet} % Default font is the helvetica postscript font
%\usepackage{newcent} % To change the default font to the new century schoolbook postscript font uncomment this line and comment the one above
\usepackage{amsmath}

\setlength{\textwidth}{5.1in} % Text width of the document
\usepackage{standalone}

\begin{document}

%----------------------------------------------------------------------------------------
%	NAME AND ADDRESS SECTION
%----------------------------------------------------------------------------------------

\moveleft.5\hoffset\centerline{\large\bf Nico Courts} % Your name at the top
 
\moveleft\hoffset\vbox{\hrule width\resumewidth height 1pt}\smallskip % Horizontal line after name; adjust line thickness by changing the '1pt'

\moveleft.5\hoffset\centerline{Graduate Student}
\moveleft.5\hoffset\centerline{University of Washington Department of Mathematics}
\moveleft.5\hoffset\centerline{nico@nicocourts.com\qquad ncourts@uw.edu\qquad github.com/NicoCourts}

%----------------------------------------------------------------------------------------

\begin{resume}

%----------------------------------------------------------------------------------------
%	RESEARCH INTERESTS SECTION
%----------------------------------------------------------------------------------------
 
\section{RESEARCH INTERESTS}  

I am a student of Prof. Julia Pevtsova studying the representation theory of algebraic groups and 
other objects using tools from geometry and cohomology theory. Looking forward, I am particularly
interested in the study of quantum groups and (non-commutative) $q$-deformations of objects in algebraic 
geometry as well as considering problems related to the geometric Langland's program.

% More applied interests
%I am a student of Prof. Julia Pevtsova studying the representation theory of group schemes and other 
%algebraic structures using tools from geometry, category theory, and homological algebra. While my thesis 
%is likely to be of a highly theoretical character, it requires that I be skilled in many different areas of math and 
%I am conversant in the languages of combinatorics, topology, analysis, and optimization.

%What's more, I am always looking for ways to ground what I am learning in the 
%real world. Applications that make me particularly excited are that of artificial general intelligence and 
%interpretability of mathematical models which, I believe, can serve a vital role in the realization of AGI.

%----------------------------------------------------------------------------------------
%	EDUCATION SECTION
%----------------------------------------------------------------------------------------

\section{EDUCATION}

{\bf Ph.D.,} Mathematics\hfill June 2022 (expected) \\
University of Washington, Seattle, WA

{\bf Master of Science,} Mathematics\hfill June 2020 (expected) \\
University of Washington, Seattle, WA \\
{\it Thesis Topic:} Schur Duality and Strict Polynomial Functors

{\bf Bachelor of Science,} Mathematics\hfill May 2016 \\
{\sl Magna Cum Laude}, Phi Beta Kappa, Dean's List, Departmental Honors \\
University of Southern California, Los Angeles, CA 

{\bf Budapest Semesters in Mathematics}\hfill Fall 2015 \\
Algebraic Topology, Conjecture \& Proof, Cryptography, Differential Geometry  \\
Budapest, Hungary 

{\bf Associate of Science,} Mathematics\hfill June 2013\\
Key of Knowledge, Dean's list, Honors Program \\
Citrus College, Glendora, CA 
 
%----------------------------------------------------------------------------------------
%	RESEARCH EXPERIENCE SECTION
%----------------------------------------------------------------------------------------

\section{RESEARCH EXPERIENCE}

{\bf Current Focus}\\
In the recent past, my studies have been focusing on understanding the rational representations of $\operatorname{GL}_n$ as established by Schur in his thesis 
and later work (and later translated and elucidated by Green). More recently, Friedlander, Suslin, Krause,
and others have found equivalent categories that more readily afford monoidal structure in the interest of establishing the monoidicity 
of the Schur-Weyl functor and examining the interactions between different forms of duality. 

My current (short-term) goal is to polish my understanding of how the phenomenon of Schur-Weyl duality presents in the broader 
context (e.g. in the relationship between $q$-Schur algebras and Iwahori-Hecke algebras) while continuing to learn about geometric methods 
used for answering questions in representation theory.

{\bf Graduate Reading Courses} \\
University of Washington, Seattle, WA
\begin{itemize} \itemsep -1pt % Reduce space between items
	\item Group Schemes and Algebraic Groups -- Prof. Julia Pevtsova -- {\sl Autumn 2018}
	\item Abelian Categories -- Prof. James Zhang -- {\sl Spring 2018}
	\item Representation Theory -- Prof. Julia Pevtsova -- {\sl Winter/Spring 2017}
	\item (Simplicial) Cohomology -- Prof. Steve Mitchell -- {\sl Autumn 2016}
\end{itemize}

\newpage
%----------------------------------------------------------------------------------------
%	PROFESSIONAL EXPERIENCE SECTION
%----------------------------------------------------------------------------------------
 
\section{TEACHING EXPERIENCE}

{\bf Graduate Teaching Assistant} \hfill Autumn 2016 -- Present \\
University of Washington, Seattle, WA

\begin{itemize} \itemsep -1pt % Reduce space between items
\item\textbf{As an instructor:} 
\begin{itemize}
	\item Math 124 -- Calculus I (Summer 2018)\vspace{5pt}

%	I put my own twist on the standard Calc I curriculum by deciding to focus on high-level understanding along with cultivating problem-solving techniques and a mathematical mindset.
%	This idea was loosely based on that of the flipped classroom; students were encouraged using quizzes to read the textbook before any material was discussed in class and large portions of contact time with instructors were dedicated to working on challenging problem sets and presenting solutions to the class.
%	Technology was used frequently and liberally as a way to both bolster understanding through visual aids and as a means to solicit frequent feedback from students to monitor their understanding and progress.\vspace{5pt}
	\item Math 308 -- Matrix Algebra (Spring 2019)\vspace{5pt}
	
%	My focus during this quarter was on developing the students' ability to solve problems they hadn't seen before. In service of this, we dedicated a third of our contact time to solving problems in small groups with my oversight.
%	My initial course structure wasn't adequately meeting the needs of the class (which I determined from soliciting feedback), so after the first midterm we pivoted to having more worked-out examples in class, for which the students were thankful.
%	In order to encourage students to investigate how linear algebra was used in the real world, the students were assigned a poster project. Topics included applications of linear algebra in medical imaging, computer graphics, and in the social sciences. 
\end{itemize}
\item \textbf{As a teaching assistant:} 
\begin{itemize}
	\item Math 120 -- Precalculus (Autumn 2017)
	\item Math 124 -- Calculus I (Winter 2017, Winter 2019)
	\item Math 125 -- Calculus II (Autumn 2016, Spring 2017)
	\item Math 126 -- Calculus III (Summer 2017, Winter 2018, Spring 2018)
	\item Math 327 -- Introductory Real Analysis (Summer 2019)
	\item Math 381 -- Discrete Mathematical Modeling (Autumn 2018)
	\item Math 403 -- Group Theory (Winter 2020)
\end{itemize}
\end{itemize}
 
{\bf Lead Teaching Assistant and Instructor} \hfill Summer 2016 \\
SCS Noonan Scholars (previously South Central Scholars), Los Angeles, CA
\begin{itemize} \itemsep -1pt
\item Independently developed and delivered  approximately 50 hours of instruction and five exams to gifted university-bound students in calculus 2 and 3.
\item Total of 100 contact hours, including daily supevised worksheet sessions.
\item Took the initiative to deliver weekly lectures in higher mathematics (number theory, knot theory, differential equations, etc.) along with entry-level problems that allowed students to get a sense of the “flavor” of these fields.
\end{itemize} 

{\bf Various Teaching and Mentorship Positions} \hfill Spring 2012 -- Summer 2013 \\
Citrus College, Glendora, CA
\begin{itemize} \itemsep -1pt
\item {\bf PAGE Program Tutor} Assisted a licensed teacher in the education of a class of middle school children intended to reinforce the previous year’s learning and to prevent “backsliding”. Personally instructed a small group of students who were prepared to learn more advanced topics in intermediate algebra.
\item {\bf SIGMA Mentor} Took on a small group of students each semester utilizing a holistic approach to education – supplementing standard tutoring with more in-depth educational guidance and planning.
\item {\bf Math Tutor} Instructed students in the fast-paced Math Success Center where I provided homework help in all math classes through linear algebra and differential equations.
\end{itemize} 

\section{LEADERSHIP \& SERVICE}

{\bf Graduate Student Representative}\\
University of Washington, Seattle\\
Summer 2019 - Spring 2020
\begin{itemize} \itemsep -1pt
	\item Planned and organized a variety of events and lectures for the graduate students as well as the department at large.
    \item Served as an advocate for the graduate students in several capacities. 
	\item Worked on promoting better communication between the students and faculty.
	\item Empowered students to make changes to the department while promoting respect for the wishes of the faculty and administration.
\end{itemize}

{\bf Washington Directed Reading Program}\\
{\sl Co-organizer and mentor}
\begin{itemize}\itemsep -1pt
	\item Ran the program along with other graduate students in the 2019/2020 year, focusing on procuring funds and encouraging participation of under-represented groups.
	\item Supervised an undergraduate student in a reading course based around Rebecca Weber's book {\sl Computability Theory} (Autumn 2018).
\end{itemize}

{\bf Math Hour Olympiad}\\
Volunteer Judge \\
University of Washington, Seattle\\
Spring 2018

{\bf Math Day}\\
Volunteer\\
University of Washington, Seattle\\
2017 and 2018

\section{\bf EVENTS ATTENDED}

{\bf Conference on Lie and Jordan Algebras and their Representations}\\
Sichuan University\\
Chengdu, Sichuan Province, P.R. China\\
January 2020

{\bf Triangulated Categories in Representation Theory and Geometry}\\
University of Sydney\\
Sydney, NSW, Australia\\
June 2019

{\bf MSRI Summer School}\\
{\sl The Mathematics of Machine Learning}\\
University of Washington, Seattle\\
July 29 - August 9, 2019

{\bf ABC Workshop}\\
{\sl Geometric and Cohomological Methods in Algebra}\\
University of Washington, Seattle\\
November 11, 2018

{\bf Joint Mathematical Meetings}\\
Seattle, WA\\
January 2016

%----------------------------------------------------------------------------------------
%	SKILLS SECTION
%----------------------------------------------------------------------------------------

\section{SKILLS \& HOBBIES} 

{\bf Languages:}
\begin{itemize} \itemsep -2pt
	\item {\sl English} -- This is my native language.
	\item {\sl German} -- Ich kann ziemlich gut Deutsch sprechen, lesen, und verstehen!
	\item {\sl Hungarian} -- Csak egy kicsit besz\'elek magyarul.
	\item {\sl Programming} -- {\bf Go}, {\bf Haskell}, Java, {\bf \LaTeX}, PHP, {\bf Python}, Typescript.
\end{itemize}
{\bf Computer Skills:} Web/Application Development, Server Administration, Sage, Windows, Linux, FreeBSD.

{\bf Life Skills:} Critical Thinking, Abstract Reasoning, Communication, Objectivity, Empathy.

{\bf Hobbies:} Hiking, Jogging, Rollerskating, Appreciating the Wonders of the PNW.



\end{resume}
\end{document}